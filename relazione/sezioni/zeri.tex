% !TEX program = lualatex
% !TEX root = ../relazioni.tex
% !TEX spellcheck = it_IT

\section{Metodi di ricerca di zeri}

\begin{esercizio}
Scrivere una funzione
\begin{center}
\lstinline£[c,fc,iter]=bisezione(f,a,b,tol,itmax)£
\end{center}
metodo di bisezione per la ricerca di una funzione \lstinline£f£ nell'intervallo \lstinline£[a, b]£. Qui, \lstinline£tol£ è la tolleranza sull'approssimazione della radice e \lstinline£itmax£ è il numero massimo di iterazioni disposte a fare. Per quanto riguarda l'output, \lstinline£c£ è l'approssimazione trovata, \lstinline£fc£ è \lstinline£f(c)£ e \lstinline£iter£ è il numero di iterazioni fatte effettivamente per arrivare all'approssimazione \lstinline£c£. Testare la funzione per la ricerca di una delle radici di
\[f(x) := e^x - x^2 - \sin x - 1\]
sull'intervallo \([-2, 2]\).
\end{esercizio}

L'implementazione qui sotto usa il \enquote{teorema degli zeri} e l'algoritmo di bisezione.

\lstinputlisting{../zeri/bisezione.m}

Testiamo il metodo di bisezione su \(f\):

\begin{lstlisting}[numbers=none]
>> f = @(x) exp(x)-x.^2-sin(x)-1;
>> df = @(x) exp(x)-2*x-cos(x);
>> [c, fc, it] = bisezione(f, 1, 1.5, 1e-12,100)
c = 1.2797
fc = 6.6813e-13
it = 39
\end{lstlisting}

% *********************************************************************************************

\begin{esercizio}
Scrivere una funzione
\begin{center}
\lstinline£[c,fc,iter]=newton(f,df,x0,tol,itmax)£
\end{center}
che implementa il metodo di \textenglish{Newton-Raphson} per la ricerca di uno zero di una funzione \lstinline£f£ con derivata \lstinline£df£. Qui, \lstinline£x0£ è il punto da cui parte il metodo, \lstinline£tol£ è la tolleranza sull'approssimazione della radice e \lstinline£itmax£ è il numero massimo di iterazioni disposte a fare. Per quanto riguarda l'output, \lstinline£c£ è l'approssimazione trovata, \lstinline£fc£ è \lstinline£f(c)£ e \lstinline£iter£ è il numero di iterazioni fatte effettivamente per arrivare all'approssimazione \lstinline£c£. Testare la funzione per la ricerca di una delle radici di
\[f(x) := e^x - x^2 - \sin x - 1\]
sull'intervallo \([-2, 2]\).
\end{esercizio}

Richiamiamo che il metodo di \textenglish{Newton-Raphson} si scrive come
\[\begin{cases} x_0 \in \erre \text{ scelto} \\ x_{n+1} = x_n - \frac{f(x_n)}{f^\prime (x_n)} \text{ per } n \in \enne \end{cases}\]

Osserviamo che se dobbiamo controllare l'incremento, ci dobbiamo arrestare appena che il valore assoluto di \(\frac{f(x_n)}{f^\prime (x_n)}\) scende al di sotto di una certa tolleranza. 

\lstinputlisting{../zeri/newton.m}

Testiamo il metodo di bisezione su \(f\):

\begin{lstlisting}[numbers=none]
>> f = @(x) exp(x)-x.^2-sin(x)-1;
>> df = @(x) exp(x)-2*x-cos(x);
>> [c, fc, it] = newton(f, df, 1.5, 1e-6,100)
c = 1.2797
fc = 2.2204e-16
it = 5
\end{lstlisting}

% *********************************************************************************************

\begin{esercizio}
Testo
\end{esercizio}

\lstinputlisting{../zeri/newton\_errors.m}
