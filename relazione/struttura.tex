% !TEX program = lualatex
% !TEX root = ./relazioni.tex

\documentclass[ structure      = article
              , maketitlestyle = standard
              , secstyle       = center
              , secfont        = roman
              %, subsecstyle    = center
              , subsecfont     = roman
              , headerstyle    = center
              , liststyle      = aligned
              , twocolcontents = toc
              ]{suftesi}

\usepackage[no-math]{fontspec}
\setmainfont{Linux Libertine O}
\setmonofont{Source Code Pro}[Scale=MatchLowercase]
%\usepackage[tt=false]{libertine}
%\usepackage{sourcecodepro}

\usepackage{polyglossia}
\setmainlanguage{italian}
\setotherlanguage{english}

\usepackage[italian=quotes]{csquotes}

\usepackage{hyperref}
\hypersetup{breaklinks,hidelinks}

\usepackage{anyfontsize}

\usepackage{indentfirst}

%\usepackage[original]{abstract}
%\renewcommand{\abstracttextfont}{\normalfont\normalsize}

\usepackage{libertinust1math}
\usepackage{MnSymbol,mathtools}
\usepackage[bb=ams]{mathalfa}
\usepackage{amsthm}
\theoremstyle{definition}
\newtheorem{esercizio}{Esercizio}[section]
\newtheorem{nota}{Nota}[section]

\usepackage{listings,xcolor}
\lstset{
    language        = Matlab
  , basicstyle      = \small\ttfamily
  , commentstyle    = \color{green!50!black}
  , keywordstyle    = \color{blue!50!black}
  %, tabsize         = 2
  , numbers         = left
  , numberstyle     = \tiny\ttfamily\color{gray!50!black}
  , numbersep       = 2mm
  , frame           = leftline
  , rulecolor       = \color{gray!50!black}
  , backgroundcolor = \color{gray!3!white}
  %, mathescape      = true
  }


\newcommand\set[1]{\left\{#1\right\}}
\newcommand\abs[1]{\left\lVert #1 \right\rVert}
\newcommand\erre{\mathbb R}
\newcommand\enne{\mathbb N}
